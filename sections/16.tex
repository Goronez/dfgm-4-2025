\documentclass[../cdg-all.tex]{subfiles}
\begin{document}
\section{Деривационные формулы Гаусса — Вейнгартена. Теорема единственности гиперповерхности с заданными
первой и второй фундаментальными формами.}
\begin{theorem}[Деривационные формулы Г-В]
 \begin{equation*}
   \left\{
  \begin{aligned}
    & \frac{\partial r_i}{\partial x^j} = \Gamma_{ij}^k r_k + b_{ij}m \\ 
    & \frac{\partial m}{\partial x^j} = -\beta_j^k r_k, \beta_j^k = -g^{ik}b_{ij}
  \end{aligned}
   \right.
 \end{equation*} 
\end{theorem}
\begin{proof}
  \[ 
  \frac{\partial r_s}{\partial x^i} = r_{si} = \Pi(r_{si}) + \nu_{si}, \text{где } \nu_{si}\perp T_P M
  \]
  По определению $\Pi(r_{si}) = \Gamma_{si}^k r_k$ и $\nu_{si} = (r_{si},m)m = b_{si}m$. Отсюда получаем первое равенство системы.\\
  Заметим, что $(m, r_s) = 0$. Тогда 
  \[
    (\frac{\partial m}{\partial x^i}, r_s) + (m, \frac{\partial r_s}{\partial x^i}) = \beta_i^k(r_k,r_s) + b_{is} = 0
  \]
  так как $\frac{\partial m}{\partial x^i}$ является линейной комбинацией векторов $r_k$ с какими-то коэффициентами $\beta_i^k$. 
  $(r_k,r_s) = g_{ks}$ и следовательно
  \[
    \beta_i^k = -g^{ks}b_{is}
  \]
  где $g^{ks}$ матрица обратная к $g_{ks}$. 
\end{proof}
\newpage
\end{document}

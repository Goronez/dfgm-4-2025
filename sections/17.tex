\documentclass[../cdg-all.tex]{subfiles}
\begin{document}
\section{Условие совместности системы дифференциальных уравнений. Теорема о разрешимости.}

Пусть $x\in D\subset \mathbb{R}^n, y,y_0\in\mathbb{R}^m, f_j(x,y):V\to\mathbb{R},(x_0,y_0)\in V\subset\mathbb{R}^{n+m}$ и 
\begin{equation*}
  \label{eq:17}
  \tag{*}
  \left\{
    \begin{aligned}
    & \frac{\partial y}{\partial x^j} = f_j(x,y) \\
    & y(x_0) = y_0  
    \end{aligned}
    \right.
\end{equation*}
Продифференцируем первое уравнение по $x^i$:
\begin{equation*}
  \frac{\partial^2 y}{\partial x^j\partial x^i} = \frac{\partial f_j}{\partial x^i} + \frac{\partial f_j}{\partial y^\alpha}\frac{\partial y^\alpha}{\partial x^i} = 
  \frac{\partial f_j}{\partial x^i} + \frac{\partial f_j}{\partial y^\alpha}f^\alpha_i
\end{equation*}
Аналогично для $f_i(x,y)$ и $x^j$:
\begin{equation*}
  \frac{\partial^2 y}{\partial x^i\partial x^j} = \frac{\partial f_i}{\partial x^j} + \frac{\partial f_i}{\partial y^\alpha}\frac{\partial y^\alpha}{\partial x^j} = 
  \frac{\partial f_i}{\partial x^j} + \frac{\partial f_i}{\partial y^\alpha}f^\alpha_j
\end{equation*}

\begin{definition}
  Система уравнений $(\ref{eq:17})$ удовлетворяет в $V$ условиям совместности, если 
  $\frac{\partial f_j}{\partial x^i} +\frac{\partial f_j}{\partial y^\alpha}f_i^\alpha=\frac{\partial f_i}{\partial x^j}+\frac{\partial f_i}{\partial y^\alpha}f_j^\alpha$. 
\end{definition}

\begin{theorem}
  Пусть $(\ref{eq:17})$ удовлетворяет в $V$ условиям совместности. Тогда в некоторой окрестности точки $x_0$ $\exists!$ гладкая функция $y(x)$, удовлетворяющая $(\ref{eq:17})$
  и условию $y(x_0) = y_0$.
\end{theorem}
\begin{proof}
  Пусть 
  \begin{equation}
    \label{eq:17.1}
    \left\{
    \begin{aligned}
      & \frac{\partial y^{(1)}}{\partial x^1} = f_1(x^1, x_0^2, \ldots, x_0^n, y^{(1)}) \\
      & y^{(1)}|_{x^1=x_0^1} = y_0
    \end{aligned}
    \right.
  \end{equation}
  По теореме единственности у данной задачи Коши есть единственное решение $y^{(1)}(x^1,x_0^2,\ldots, x_0^n)$ на некотором интервале.
  Рассмотрим систему 
  \begin{equation}
    \label{eq:17.2}
    \left\{
    \begin{aligned}
      & \frac{\partial y^{(2)}}{\partial x^2} = f_2(x^1, x^2,x_0^3 \ldots, x_0^n, y^{(1)}) \\
      & y^{(2)}|_{x^2=x_0^2} = y^{(1)}(x^1, x_0^2, \ldots, x_0^n)
    \end{aligned}
    \right.
  \end{equation}
  Правые части системы $(\ref{eq:17.2})$ зависят от $x^1$ как от параметра. Пользуясь теоремой единственности и теоремой о гладкой зависимости от параметра, находим
  функцию $y^{(2)}(x^1, x^2, x_0^3, \ldots, x_0^n)$, удовлетворяющую системе $(\ref{eq:17.2})$.
  Но данная функция также удовлетворяет системе $(\ref{eq:17.1})$ поскольку:
  \[
    g = \frac{\partial y^{(2)}}{\partial x^1} - f_1(x^1, x^2, x_0^3,\ldots,x_0^n, y^{(2)})
  \]
\begin{equation*} 
  \label{eq:17.3}
  \frac{\partial g}{\partial x^2} = \frac{\partial^2 y^{(2)}}{\partial x^1\partial x^2} - \frac{\partial f_1}{\partial x^2} - 
  \frac{\partial f_1}{\partial y^\alpha}\frac{\partial y^{(2)\alpha}}{\partial x^2} = 
  \frac{\partial f_2}{\partial x^1} +\frac{\partial f_2}{\partial y^\alpha}\frac{\partial y^{(2)\alpha}}{\partial x^1} - \frac{\partial f_1}{\partial x^2} - \frac{\partial f_1}{\partial y^\alpha}f_2^\alpha \tag{**}
\end{equation*}
Из условий совместности следует, что
$\frac{\partial f_2}{\partial x^1} -\frac{\partial f_1}{\partial x^2} =\frac{\partial f_1}{\partial y^\alpha}f_2^\alpha - \frac{\partial f_2}{\partial y^\alpha}f_1^\alpha$. 
Тогда
  \begin{equation*}
  (\ref{eq:17.3}) = 
   \frac{\partial f_1}{\partial y^\alpha}f_2^\alpha - \frac{\partial f_2}{\partial y^\alpha}f_1^\alpha +\frac{\partial f_2}{\partial y^\alpha}\frac{\partial y^{(2)\alpha}}{\partial x^1} - \frac{\partial f_1}{\partial y^\alpha}f_2^\alpha = 
    \frac{\partial f_2}{\partial y^\alpha}(\frac{\partial y^{(2)\alpha}}{\partial x^1} -f_1^\alpha) 
    \Rightarrow \frac{\partial g}{\partial x^2} = \frac{\partial f_2}{\partial y^\alpha}g^\alpha
\end{equation*}
  При $g|_{x^2 = x_0^2} = 0$. Поэтому по теоерме единственности система
  \begin{equation*}
    \left\{
    \begin{aligned}
      & \frac{\partial g}{\partial x^2} = \frac{\partial f_2}{\partial y^\alpha}g^\alpha \\
      & g|_{x^2 = x_0^2} = 0
    \end{aligned} 
  \right.
  \end{equation*} 
имеет решение $g \equiv 0$ в некоторой окрестности точек $x_0^1, x_0^2$.
  Аналогично для функций $y^{(k)}, f_k$, где $k = 3,\ldots, n$. В результате получаем функцию $y(x) = y^{(n)}(x)$, удовлетворяющую всем $n$ системам. \\
  Так как 
  \begin{align*}
    y|_{x^n = x_0^n} &= \,y^{(n - 1)}(x^1,\ldots,x^{n-1}) \\
    &\shortvdotswithin{=}
    y|_{x^n = x_0^n,\ldots,x^2 = x_0^2} &= y^{(1)}(x^1, x_0^2,\ldots,x_0^n) \\
    y|_{x^n = x_0^n,\ldots,x^1 = x_0^1} &= y^{(1)}(x_0^1, x_0^2,\ldots,x_0^n) = y^{(1)}|_{x^1=x_0^1}=y_0
  \end{align*}
  то $y(x)$ удовлетворяет начальному условию. \\ 
  Докажем единственность: пусть в некоторой окрестности $x_0$ $y(x)$ -- решение $(\ref{eq:17})$ . Тогда рассмотрим функцию $y(x^1, x_0^2,\ldots, x_0^n)$, 
  но это в точности $y^{(1)}$, так как она удовлетворяет системе $(\ref{eq:17.1})$, а у неё в свою очередь есть единственное решение.
  Далее: $y(x^1, x^2, x_0^3,\ldots, x_0^n) = y^{(2)}$ поскольку она удовлетворяет $(\ref{eq:17.2})$ и $(\ref{eq:17.1})$ при $x^2 = x_0^2$. Таким образом в итоге получаем, что
  $y(x^1,\ldots,x^n) = y^{(n)}$.
\end{proof}
\newpage
\end{document}

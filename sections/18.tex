\documentclass[../cdg-all.tex]{subfiles}
\begin{document}
\section{Условия совместности для системы Гаусса-Вейнгартена. Равенства Гаусса и Петерсона-Майнарди-Кодацци.
Блистательная теорема Гаусса}
Условия совместности для уравнений Г и ПМК:
\begin{multline*}
  %\label{eq:g}
  \frac{\partial \Gamma_{sj}^k}{\partial x^i}r_k + \frac{\partial r_l}{\partial x^i}\Gamma_{sj}^l + \frac{\partial b_{sj}}{\partial x^i}m + \frac{\partial m}{\partial x^i}b_{sj}=
  \frac{\partial \Gamma_{sj}^k}{\partial x^i}r_k + (\Gamma_{li}^kr_k+b_{li}m)\Gamma_{sj}^l + \frac{\partial b_{sj}}{\partial x^i}m -b_{sj}\beta_i^kr_k=\\
  (\frac{\partial \Gamma_{sj}^k}{\partial x^i}+\Gamma_{li}^k\Gamma_{sj}^l-b_{sj}\beta_i^k)r_k + (b_{li}\Gamma_{sj}^l+ \frac{\partial b_{sj}}{\partial x^i})m
\end{multline*}
Поменяем местами индексы $i,j$ и получим условия совместности:
\begin{equation}
  \label{eq:g}
  \tag{\text{Г}}
  \frac{\partial \Gamma_{sj}^k}{\partial x^i}+\Gamma_{li}^k\Gamma_{sj}^l-b_{sj}\beta_i^k = \frac{\partial \Gamma_{si}^k}{\partial x^j}+\Gamma_{lj}^k\Gamma_{si}^l-b_{si}\beta_j^k
\end{equation}
\begin{equation}
  \label{eq:pmc}
  \tag{\text{ПМК}}
  b_{li}\Gamma_{sj}^l+ \frac{\partial b_{sj}}{\partial x^i} = b_{lj}\Gamma_{si}^l+ \frac{\partial b_{si}}{\partial x^j}
\end{equation}
\begin{definition}[Уравнения Г и ПМК]
  \[
    \frac{\partial\Gamma_{sj}^k}{\partial x^i} - \frac{\partial\Gamma_{si}^k}{\partial x^j} + \Gamma_{sj}^l\Gamma_{li}^k - \Gamma_{si}^l\Gamma_{lj}^k - 
    \beta_{sj}b_i^k - \beta_{si}b_j^k = 0
  \]
  \[
    \Gamma_{sj}^{l}b_{li} - \Gamma_{si}^{l}b_{lj} + \frac{\partial b_{sj}}{\partial x^i} - \frac{\partial b_{si}}{\partial x^j} = 0
  \]
\end{definition}

\begin{theorem}
  Система Гаусса-Вейнгартена совместна $\Leftrightarrow$ выполнены равенства Гаусса и Петерсона-Майнарди-Кодацци.
\end{theorem}
\begin{proof} 
  $(\Rightarrow)$ Уравнения $\ref{eq:g}$ и $\ref{eq:pmc}$, очевидно, выполняются. 
  \begin{equation*}
    -\frac{\partial^2 m}{\partial x^j \partial x^i} = \frac{\partial \beta_j^k}{\partial x^i}r_k + \beta_j^l\frac{\partial r_l}{\partial x^i} = 
    \frac{\partial \beta_j^k}{\partial x^i}r_k + \beta_j^l(\Gamma_{li}^k r_k + b_{li}m) = 
    (\frac{\partial \beta_j^k}{\partial x^i}+\beta_j^l\Gamma_{li}^k)r_k +(\beta_j^l b_{li})m
  \end{equation*}
  Тогда условия совместности для $\frac{\partial^2 m}{\partial x^j \partial x^i}$:
  \begin{equation}
    \label{eq:a}
    \tag{a}
    \frac{\partial \beta_j^k}{\partial x^i}+\beta_j^l\Gamma_{li}^k = \frac{\partial \beta_i^k}{\partial x^j}+\beta_i^l\Gamma_{lj}^k
  \end{equation}
  \begin{equation} 
    \tag{b}
    \label{eq:b}
    \beta_j^l b_{li} = \beta_i^l b_{lj}
  \end{equation}
  Уравнение (\ref{eq:b}) выполняется автоматически, так как 
  \[
  g^{sl}b_{sj}b_{li} - g^{sl}b_{si}b_{lj} = g^{sl}b_{sj}b_{li} - g^{ls}b_{li}b_{sj} = 0
  \]
  А уравнение (\ref{eq:a}) выполняется, если выполняется \ref{eq:pmc}
  
     
  
\end{proof}

\newpage
\end{document}
